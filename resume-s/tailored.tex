\documentclass[letterpaper,11pt]{article}

% --- STANDARD PACKAGE IMPORTS (Leave these as-is) ---
\usepackage{latexsym}
\usepackage[empty]{fullpage}
\usepackage{titlesec}
\usepackage{marvosym}
\usepackage[usenames,dvipsnames]{color}
\usepackage{verbatim}
\usepackage{enumitem}
\usepackage[hidelinks]{hyperref}
\usepackage{fancyhdr}
\usepackage[english]{babel}
\usepackage{tabularx}
\usepackage{fontawesome5}
\usepackage{multicol}
\setlength{\multicolsep}{-3.0pt}
\setlength{\columnsep}{-1pt}
\input{glyphtounicode}
\usepackage[margin=1.4cm]{geometry}


% --- FORMATTING & STYLES (Leave these as-is) ---
\pagestyle{fancy}
\fancyhf{} % clear all header and footer fields
\fancyfoot{}
\renewcommand{\headrulewidth}{0pt}
\renewcommand{\footrulewidth}{0pt}

% Adjust margins
\addtolength{\oddsidemargin}{-0.15in}
 \addtolength{\textwidth}{0.3in}

\urlstyle{same}

\raggedbottom
\raggedright
\setlength{\tabcolsep}{0in}

% Sections formatting
\titleformat{\section}{
  \vspace{-4pt}\scshape\raggedright\large\bfseries
}{}{0em}{}[\color{black}\titlerule \vspace{-5pt}]

% Ensure that generate pdf is machine readable/ATS parsable
\pdfgentounicode=1

%-------------------------
% Custom commands (These define the look of the resume - no changes needed here)
\newcommand{\resumeItem}[1]{
  \item\small{
    {#1 \vspace{0pt}}
  }
}

\newcommand{\classesList}[4]{
    \item\small{
      {#1 #2 #3 #4 \vspace{-2pt}}
  }
}

\newcommand{\resumeSubheading}[4]{
  \vspace{-2pt}\item
    \begin{tabular*}{1.0\textwidth}[t]{l@{\extracolsep{\fill}}r}
      \textbf{#1} & \textbf{\small #2} \\
      \textit{\small#3} & \textit{\small #4} \\
    \end{tabular*}\vspace{-7pt}
}

\newcommand{\resumeSubSubheading}[2]{
    \item
    \begin{tabular*}{0.97\textwidth}{l@{\extracolsep{\fill}}r}
      \textit{\small#1} & \textit{\small #2} \\
    \end{tabular*}\vspace{-7pt}
}

\newcommand{\resumeProjectHeading}[2]{
    \item
    \begin{tabular*}{1.001\textwidth}{l@{\extracolsep{\fill}}r}
      \small#1 & \textbf{\small #2}\\
    \end{tabular*}\vspace{-7pt}
}

\newcommand{\resumeSubItem}[1]{\resumeItem{#1}\vspace{-4pt}}

\renewcommand\labelitemi{$\vcenter{\hbox{\tiny$\bullet$}}$}
\renewcommand\labelitemii{$\vcenter{\hbox{\tiny$\bullet$}}$}

\newcommand{\resumeSubHeadingListStart}{\begin{itemize}[leftmargin=0.0in, label={}]}
\newcommand{\resumeSubHeadingListEnd}{\end{itemize}}\vspace{0pt}
\newcommand{\resumeItemListStart}{\begin{itemize}}
\newcommand{\resumeItemListEnd}{\end{itemize}\vspace{-5pt}}


\begin{document}

%----------HEADING - YOUR CONTACT INFORMATION----------
\begin{center}
    {\Large \scshape Yash Bhardwaj} \\[3mm]
    \footnotesize \raisebox{-0.1\height}
    \faPhone\ \underline{9xxxxxxxxx} ~ 
    {\faEnvelope\  \underline{sample@gmail.com}} ~ 
    % REPLACE [LinkedIn URL] and [Your Handle] with your actual info
    {\faLinkedin\ \underline{\href{https://linkedin.com/in/random-user}{linkedin.com/in/random-user}} ~
    % REPLACE [GitHub URL] and [Your Handle] with your actual info
    {\faGithub\ \underline{\href{https://github.com/bhardwaj-yash1}{github.com/bhardwaj-yash1}} ~
    % Optional: Uncomment the line below to add a personal portfolio link
    % {\faBriefcase\ \underline{\href{[Portfolio URL]}{[Your Portfolio Handle]}}
    \vspace{-5pt}
\end{center}

%-----------PROFESSIONAL SUMMARY-----------
\section{Professional Summary} \\[1mm]
    % Write a brief 3-4 line summary tailored to the job (e.g., Software Developer Intern).
    % Mention your top skills (Python, React, Docker) and career goals.
    {Computer Engineering student with proven skills in full-stack web development, including React.js, JavaScript, and backend technologies like FastAPI. Experienced in building interactive applications, dashboards, and user-friendly interfaces for data visualization and real-time interactions. Seeking a full-stack internship to contribute to engaging online content and educational tools while developing skills in interactive web technologies.}
    
    \vspace{-7pt}
%-----------EDUCATION-----------
\section{Education} \\[1mm]
 \resumeSubHeadingListStart
    \resumeSubheading
      {cooked college of information technology}{|Expected Graduation: Aug 2027}
      {Bachelor of Computer Engineering | Minor Artificial Intelligence}
      {Bhubaneswar, India}
 \resumeSubHeadingListEnd
    \resumeItemListStart
        \resumeItem {GPA: 7.86/10}
        \vspace{-7pt}
        % List relevant technical courses here.
        \resumeItem {Courses: Data Structure and Algorithms, Object Oriented Programming, DBMS, Computer Architecture, Linear Algebra}
    \resumeItemListEnd
    \vspace{-22pt}

%-----------EXPERIENCE (Optional: Uncomment to use)---------------
% \section{Work Experience}
%     \resumeSubHeadingListStart
%             \resumeSubheading{[Company Name]}{[Start Month Year] -- [End Month Year]}{[Job Title]}{[Location]} 
%             \resumeItemListStart
%                     \resumeItem{Quantifiable achievement 1 (Use action verbs and metrics: e.g., Engineered LLMs, improving efficiency by 37\%).}
%                     \resumeItem{Technical detail 2 (Specify tools/tech: e.g., Analyzed 1200+ reqs using scikit-learn, saving company \$20,000).}
%                     \resumeItem{Project detail 3 (Focus on implementation: e.g., Integrated TF-IDF, optimizing NLP analysis processes).}
%             \resumeItemListEnd
%     \resumeSubHeadingListEnd
%     \vspace{-12pt}

%-----------PROJECTS-----------
\section{Projects} 
    \vspace{-5pt}
    \resumeSubHeadingListStart
    
    % --- Project 1 ---
    \resumeProjectHeading
         % Project Name, Linked Source Code
         {\textbf{DriftRadar} $|$ \emph{\href{https://github.com/bhardwaj-yash1}{Source Code}}}{ReactJS $|$ Docker $|$ FastAPI $|$ PostgreSQL}
         \\[5mm]
          \resumeItemListStart
             % Detail 1: What did you DO and what was the RESULT? (Action verb: Developed...)
             \resumeItem{Developed a full-stack MLOps platform to detect and monitor data/model drift in real time, creating an interactive dashboard for user engagement.}
             % Detail 2: What TECHNOLOGIES did you use for the core logic? (Action verb: Built...)
             \resumeItem{Built a FastAPI REST backend with database integration using PostgreSQL and a React.js frontend for visualization and interactive features.}
             % Detail 3: What about DEPLOYMENT/PROCESS? (Action verb: Containerized...)
             \resumeItem{Containerized the application with Docker for scalable deployment, enabling troubleshooting and alignment with CI/CD best practices.}
          \resumeItemListEnd
           \vspace{-20pt}
    
    % --- Project 2 ---
    \resumeProjectHeading
          % Project Name, Linked Source Code
          {\textbf{Animal Disease Detection Model} $|$ \emph{\href{https://github.com/bhardwaj-yash1}{Source Code}}}{Python $|$ Scikit-learn $|$ Pandas $|$ Streamlit}
          \\[5mm]
          \resumeItemListStart
             \resumeItem{Implemented a Python-based web application using Streamlit for user-friendly interfaces to display interpretable predictions.}
             \resumeItem{Applied data preprocessing and feature engineering with Pandas and Scikit-learn, integrating with models for efficient analysis.}
             \resumeItem{Deployed the application to support interactive queries and visualizations, focusing on maintainable code and documentation.}
          \resumeItemListEnd
    
    % --- Project 3 ---
    \vspace{-17pt}
    \resumeProjectHeading
    % Project Name, Linked Source Code
    {\textbf{FAQ Bot for Discord} $|$ \emph{\href{https://github.com/bhardwaj-yash1}{Source Code}}}{Python $|$ Scikit-learn $|$ Discord.py $|$ NLTK}
    \\[5mm]
    \resumeItemListStart
    \resumeItem{Designed a backend service in Python with NLTK for natural language processing and real-time query handling in interactive environments.}
    \resumeItem{Trained models using Scikit-learn for intent classification, integrating with Discord.py for real-time data processing.}
    \resumeItem{Developed automated response systems with debugging and code review practices to ensure reliable performance in user interactions.}
    
    \resumeItemListEnd

\resumeSubHeadingListEnd
\vspace{-20pt}
%-----------LEADERSHIP & EXTRACURRICULARS-----------
\section{Leadership \& Extracurriculars}
    \vspace{-3pt}
    \resumeSubHeadingListStart

        % --- Activity 1 ---
        \resumeProjectHeading
            {\textbf{Film and Theater Society, IIIT Bhubaneswar}}
            {Aug 2023 – May 2025}
        \resumeItemListStart
            \resumeItem{Edited and produced 15+ event promotion videos, boosting engagement by 30\%.}
            \resumeItem{Led a 4-member media team for the annual fest, managing live coverage and media production.}
        \resumeItemListEnd
        \vspace{-17pt}
        
        % --- Activity 2 ---
        \resumeProjectHeading
            {\textbf{Advaita Techno-Cultural Fest, IIIT Bhubaneswar}}
            {Core Team}
        \resumeItemListStart
            \resumeItem{Coordinated logistics for 10+ events, ensuring smooth execution.}
            \resumeItem{Organized Inter-College Robo Soccer Competition with 100+ participants.}
        \resumeItemListEnd
        
    \resumeSubHeadingListEnd
\section{Technical Skills}
\vspace{-4pt}
{ % This defines the table structure for your skills section
\renewcommand{\arraystretch}{1.4} % Adjusts vertical spacing between skill lines
\begin{tabularx}{\linewidth}{@{} p{5.4cm} X @{}} % Defines two columns, the first is fixed width, the second auto-fills
    \textbf{Programming \& Databases} & Python, SQL, C++, JavaScript, PostgreSQL \\
    \textbf{Web Development} & React.js, JavaScript, Streamlit, FastAPI, REST API \\
    \textbf{Data Science \& Analytics} & Pandas, NumPy, Scikit-learn, Exploratory Data Analysis (EDA), Data Visualization (Matplotlib, Seaborn) \\
    \textbf{Deployment \& Tools} & Docker, Git, GitHub, Jupyter, VS Code \\
    % Add other categories as needed (e.g., NLP, Cloud, Testing)
\end{tabularx}
}

% --- SPACING (Leave these as-is) ---
 \vspace{-16pt}
 \vspace{3pt}
\vspace{10pt}

\vspace{-15pt}


\end{document}